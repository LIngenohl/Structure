\documentclass[12pt]{article}
	\usepackage[T1]{fontenc}
	\usepackage[utf8]{inputenc}
	\usepackage[british]{babel}
	\usepackage[a4paper]{geometry}
	\geometry{verbose,tmargin=3cm,bmargin=3cm,lmargin=2cm,rmargin=2cm,marginparwidth=70pt}
	\setcounter{secnumdepth}{3}
	\setcounter{tocdepth}{3}
	\setlength{\parindent}{4em}
	\setlength{\parskip}{1em}
	\renewcommand{\baselinestretch}{1.5}
	\usepackage{prettyref}
	\usepackage{textcomp}
	\usepackage{setspace}
	\usepackage{indentfirst}
	\usepackage{fancyhdr}
	\usepackage{url}
	\usepackage[normalem]{ulem}
	\usepackage[table, fixpdftex]{xcolor}
	\usepackage{algpseudocode}
	\usepackage{bigstrut}
	\usepackage{enumitem}

	% package hyperref
    \usepackage[hidelinks]{hyperref}
  

	% biblatex
	\usepackage[style=authoryear,natbib=true,maxcitenames=2, maxbibnames=11,backend=biber,pagetracker=page,hyperref=true]{biblatex} \usepackage{csquotes}
	\renewcommand*{\bibsetup}{%
		\interlinepenalty=10000\relax % default is 5000
		\widowpenalty=10000\relax
		\clubpenalty=10000\relax
		\raggedbottom
		\frenchspacing
        \biburlsetup}
        
	% fixes the page number of the first page of each chapter
	\fancypagestyle{plain}{
			\fancyhead{}
			\renewcommand{\headrulewidth}{0pt}
			\renewcommand{\footrulewidth}{0pt}
			\fancyfoot[OC]{\begin{flushright}\thepage\end{flushright}}
    }
    
	% fancy headers for the thesis
	\fancyhead{}
	\fancyhead[RO]{\slshape \nouppercase \rightmark}
	\fancyfoot[OC]{\begin{flushright}\thepage\end{flushright}}
	\renewcommand{\headrulewidth}{0.4pt}
	\setlength{\headheight}{14pt}

	% add bibliography database
	\addbibresource{BA Kopie.bib}
	
	% space between biblio items
	\setlength\bibitemsep{1.7\itemsep} 
	
	% title without ""
	\DeclareFieldFormat[inbook]{title}{#1}
	% non-italic
	\DeclareFieldFormat[online]{tlaitle}{#1} 
	% title unquoted
	\DeclareFieldFormat[article]{title}{#1} 
	% no pp. 
	\DeclareFieldFormat[article]{pages}{#1} 
	% bold volume
	\DeclareFieldFormat*{volume}{\mkbibbold{#1}\setpunctfont{\textbf}}
	
	% no in:
	\renewbibmacro{in:}{} 
	
	% (volume)
	\renewbibmacro*{volume+number+eid}{%
			\printfield{volume}%
			%\setunit*{\adddot}% DELETED
			% \setunit*{\addnbspace}% NEW (optional); there's also \addnbthinspace
			\printfield{number}%
			% \setunit{\addcomma\space}%
			\printfield{eid}}
	\DeclareFieldFormat[article]{number}{\mkbibparens{#1}} 
	
	% edition.
	\DeclareFieldFormat{edition}%
	{(\ifinteger{#1}%
			{\mkbibordedition{#1}\addthinspace{}ed.}%
			{#1\isdot}).}
	
	% publisher and location position
	\renewbibmacro*{publisher+location+date}{%
			\printlist{publisher}%
			\setunit*{\addcomma\space}%
			\printlist{location}%
			\setunit*{\addcomma\space}%
			\usebibmacro{date}%
			\newunit}
	
	% shortauthor before author
	\renewbibmacro*{begentry}{%
			\ifkeyword{Key}{\sffamily}{}%
			\iffieldundef{shorthand}
			{}
			{\global\undef\bbx@lasthash
					\printfield{shorthand}%
					\addcolon\space}%
			\ifboolexpr{test {\usebibmacro{bbx:dashcheck}} or test {\ifnameundef{shortauthor}}}%
			{}%
			{\printnames{shortauthor}%
                    \addspace\textendash\space}}
                    
\title{Structure}
\author{Leopold Ingenohl}


\begin{document}
\maketitle

\section{Introduction}

\subsection{Anecdotal Reference}
\begin{center} 
Objective: highlight the importance of the financial condition -- Background
\end{center}

    \begin{enumerate}
		\item HNA
		\item China’s largest private conglomerate which over the past few years invested around \$40bn in businesses around the world 
        \item "The financial condition of investors is often in the center of financial journalism"
        \item build up of around 9.9\% stake of around \$4bn in Deutsche Bank in 2017 
        \item just below the 10\% threshold above which stake purchases must be approved by Germany’s financial watchdog 
		\item Deutsch Stock down nearly 14\% in one year
		\item cut down its stake to 8.8\%
		
        \item Financing of the group has come under strain as a result of an official crackdown on risky financing at acquisitive private enterprises in China 
		\item Liquidity issues 
		\item Potential cash-shortfall
        \item Heavily leveraged group
        \item drawing scrutiny for the complex financing
		\item S\&P global rating downgrade citing a „deteriorating liquidity profile“
		
    \end{enumerate}

\subsection{Minority Acquisitions by Corporations} 
\begin{center}
Objective: Why do Companies make minority Acquisitions? -- Background
\end{center} 
Topics: Characteristics | Motivation | Reasoning 
    \begin{enumerate}

		\item Minority acquisitions are an important organizational choice, accounting for around 20\% of all acquisitions between 1990 and 2015.\citep{Huang2017}

		\item Most firms, private and public, have assets on their books that can be considered to be non-operating assets. The second is investments in equities and bonds of other firms, sometimes for investment reasons and sometimes for strategic ones \citep{Damodaran2005}
		
        \item Minority acquisitions tend to involve the transfer of a sizable portion of the target firm, with a mean purchase of 12\%. \citep{Ouimet2013}
		
		\item A toehold often likely opens the door to a more intensive cooperation between the toeholder and the target \citep{Povel2014}
		\item Taking a toehold allows the potential acquirer to interact with the target and its management \citep{Povel2014}
	   	\item Our results suggest that bidders use minority acquisitions when they confront informational or integration barriers. \citep{Huang2017}
	   
		\item Minority acquisitions can also help to serve as a stepping-stone towards full control. \citep{Huang2017}
		\item the possibility that business agreements, alliances, or joint ventures might be reached between target firms and corporate owners. \citep{Allen2000}
		
        \item A toehold is defined as the raider’s ownership stake in the target firm prior to announcing his tender offer. SEC regulation specifies that anyone who acquires 5\%of a company’s outstanding shares must file Schedule 13(D) within ten days to disclose their identity, the number of shares owned, and their purpose in acquiring the shares. In practice, a bidder can continue purchasing target shares anonymously after hitting the 5\% threshold until the disclosure date. \citep{Goldman2005}

    \end{enumerate}

\subsection{SC 13D Filings - Toeholds \& Acquisitions} 
\begin{center}
Objective: Why do they behave activist? Active vs. Passive -- Background
\end{center}
Topics: Acquisitions | Friendly Takeovers | Hostile Takeovers | Joint-Venture | Activism | Comparison HF

    \begin{enumerate}
        \item Schedule 13D filings must be made within 10 days of acquiring a beneficial ownership of 5\% or greater of the outstanding common stock of a U.S. public company. \citep{Brigida2012}
        \item Filing a Schedule 13D allows the investor to behave in an active manner. \citep{Brigida2012}  
        \item starting with Mikkelson and Ruback (1985), note that mergers and takeovers are often preceded by the acquisition of a minority stake in the target. \citep{Greenwood2009}
        \item Implication of intention of possible merger and takeover, means they have to file as activists

        \item Definition activist HF as example -- In the spirit of Pound (1992), we define an entrepreneurial activist as an investor who buys a large stake in a publicly held corporation with the intention to bring about change and thereby realize a profit on the investment \citep{Klein2009}

        \item If: Acquisitions / friendly takeovers, hostile takeovers
        \item Acquire a minority position to access greater information about the target firm and better assess the potential for a majority acquisition. \citep{Ouimet2013}
        \item Minority acquisitions can also help to serve as a stepping-stone towards full control. \citep{huang}
        \item A toehold is defined as the raider’s ownership stake in the target firm prior to announcing his tender offer. \citep{Goldman2005}
        \item makes the toehold bidder a more aggressive competitor in the presence of rivals. \citep{Mitchell2011}
    \end{enumerate}

\subsection{Abnormal Returns for the Subject}
\begin{center}
Objective: Why are corporations interesting? -- Problem Statement
\end{center}
Topics: Comparison HF | Source of Runup | Corporate Activism 
    \begin{enumerate}
        \item see Thesis.
        
        \item % Comparison HF
        In recent studies of what happens to the target's stock  \citet{Collin-Dufresne2015} observed a positive significant market reaction upon a more general sample of Schedule 13D filings inculding all investor types. \citet{Brav2008} have shown a favorable market reaction -- 7\%-8\% average abnormal returns in the (-20|20) event window -- particularly to Schedule 13D's filed by hedge funds. Similar results have been shown by \citet{Klein2009} who observe 10.2\% average abnormal stock returns specifically for hedge fund targets.
        % Corporate Activism 
        Furthermore the runup is even higher if the acquirer is a private investor or a non-financial corporation \citep{Brigida2012}. This is matching with \citet{Akhigbe2007} findings who observed greater gains for the target's stock if the partial position was initiated by a corporate bidder.
        Concluding, all filings are followed by positive market reactions however, those submitted by corporations seem to have a stronger impact.

        \item is still largely unanswered where the announcement premium (and the upward drift in stock prices thereafter, for that matter) comes from \citep{Greenwood2009}
        \item \citet{Greenwood2009} address this issue by presuming that the runup is a reflection of investors' expectations of the target firm being acquired at a premium to the current stock price.

    \end{enumerate}


\subsection{Corporate Investments \& Corporate Condition}
\begin{center}
Objective: Hypothesis -- Research Aim 
\end{center}
Topics: Abnormal Returns | Financial Condition | Activism | Hypothesis | Conclusion
    \begin{enumerate}
        \item Abnormal returns -- Company Condition 
        \item Bridge: As they file SC13D's they behave in an activist manner which is in the sense of \citet{Klein2009} the intention to bring about change
        \item Hypotheses: In order to be able to bring change, the filer/activist corporation needs to be in a sufficient condition 
    
        \item Especially in the case of acquisition: By giving these filings a higher probability of acquisition and tracing the runups back to this assumption, the financial condition of the investor -- in order to carry out a possible acquisition -- should play an important role. For the reason that a strong investor could increase the likelihood of takeover and hence explain the strong abnormal returns.

        \item Based on these findings, the economic significance of activist minority acquisitions by corporation is apparent and the link between the financial condition of the investor and the subsequent abnormal returns on the target stock is an interesting issue to examine. 
    \end{enumerate}

\subsection{Procedure}
\begin{center}
Objective: What is the structure of the paper?
\end{center}

   



\section{Literature Review}

\subsection{Schedule 13(D)}
\begin{center}
Objective: Contextualizing the Thesis
\end{center}
Topics: Historical Background | Information contained

	\begin{enumerate}


		\item Schedule 13D filings must be made within 10 days of acquiring a beneficial ownership of 5\% or greater of the outstanding common stock of a U.S. public company. The use of the qualifier ‘beneficial’ is important because related, yet different entities, may have to file a schedule 13D if their combined ownership of the target is 5\% or greater and their voting or investment power is combined \citep{Brigida2012}

		\item Those investors with activist intentions must file a more detailed Schedule 13D, which along with other information, requires the investor to state its future intentions with respect to influencing control of the company \citep{Giglia2018}

		\item Exchange Act of 1934 (1934 Act)16 in an attempt to increase regulation of tender offers and accumulations of stock. There were no corresponding regulations in connection with cash tender offer \citep{Giglia2018}.

        \item Of relevance here is section 13(d), which governs disclosures of beneficial ownership interests in excess of five percent of certain classes of equity securities. \citep{Giglia2018}

		\item Purpose of the filing -- Instead, the purpose of the section focused on informing investors about purchases of large blocks of shares acquired in a short period of time by individuals who could then influence or change control of the issuing company \citep{Giglia2018}
		
		\item Within the Schedule 13D and 13G filings is information important to this analysis. \citep{Brigida2012}
	\end{enumerate}

\subsection{Institutional Investor Activism}
\begin{center}
Objective: What are the findings so far?
\end{center}
Topics: Characteristics AR | Problems in Comparison | Motivation of HF | Activism

	\begin{enumerate}
		\item Short-horizon event studies of stock returns: Many studies have examined what happens to targets firm’s stock price when there is a Schedule 13D filing with the SEC \citep{CoffeeJr.2014}

		\item U.S. Activism Events 

		\item Hedge fund activism and proxy fights lie between these two extremes in the “congealing of share votes,” as they are associated with toehold investments by the activist that average 8.8\%and 9.9\%, and are associated with average valuation effects of 5.0\%and 6.8\%, respectively.\citep{Denes2017}

		\item Specifically, hedge fund targets earn 10.2\% average abnormal stock returns during the period surrounding the initial Schedule 13D. Other activist targets experience a significantly positive average abnormal return of 5.1\% around the SEC filing window \citep{Klein2009}

		\item We find that the market reacts favorably to activism, consistent with the view that it creates value. The filing of a Schedule 13D revealing an activist fund’s investment in a target firm results in large positive average abnormal returns, in the range of 7\% to 8\%, during the (–20,+20) announcement window \citep{Brav2008}

		\item The average announcement return, over all of these events, is 2.36\%, about half of that earned by firms that are eventually taken over. \citep{Greenwood2009}

	
		\item  1. The first objective includes events in which the hedge fund believes that the company is undervalued and/or that the fund can help the manager maximize shareholder value.

        \item 2. The second category, which represents 17.4\% of the full sample, includes activism targeting firm’s payout policy and capital structure.

        \item 3. The third set of events includes activism targeting issues related to business strategy, such as operational efficiency, business restructuring, mergers and acquisitions, and growth strategies.

        \item 4. The fourth category of activist events involves activism urging the sale of the target.

		\item 5. Lastly, the fifth set of events includes activism targeting corporate governance.
		
		\item Greenwood: Market Return Model, matching portfolios and CAR; Brav only B\&H return VW market index; Klein B\&H returns but adjusted 
	\end{enumerate}

\subsection{Corporate 13(D) Filings}
\begin{center}
	Objective: What do they trigger? 
\end{center}
Topics: Results (Abnormal Returns) | Explanations 
\end{document}

	\begin{enumerate}
		\item but the runup is even larger if the acquirer is a nonfinancial corporation or a private investor.\citep{Brigida2012}

		\item Third, among 13D filings, the level of informed trading is higher when the filer is a nonfinancial corporation, private investment firm, intends to merge or acquire, or intends to be an activist investor \citep{Brigida2012}
	
		\item Market-adjusted returns (eret) are higher on days when Schedule 13D filers trade. \citep{Collin-Dufresne2015}
	
		\item At the other extreme, corporate takeovers typically involve the formation of large blockholdings and create large changes in firm valuation that average 15.3\% \citep{Denes2017}

		\item The runup reflects takeover rumors generated from various public sources, such as Schedule 13(d) filings with SEC disclosing stake purchases of 5\% or more in the target, media speculations, and street talk \citep{Mitchell2011}

		\item Partial bids initiated by corporate bidders are more likely to result in a full acquisition, and the size of the acquired stake and the level of institutional ownership are positively linked to the probability of acquisition. \citep{Akhigbe2007}

		\item Without exception, BIDCORP is positive and significant. Partial positions taken by corporate bidders (BIDCORP) generate significantly higher gains to the PTs. This result may reflect the hubris-based view (Roll, 1986) that corporate bidders are likely to overpay in the event of a full takeover. \citep{Akhigbe2007}

		\item Within the sample of 13D filings, some of the acquirers are corporations that are potential fullacquirers, while other acquirers are institutional investors that are not likely to pursue a complete takeover . \citep{Brigida2012}

	\end{enumerate}

\subsection{Minority Acquisitions}
\begin{center}
	Objective: Why do Corporations invest in others? 
\end{center}
Topics: General Objectives | Definitions | Motivation | Minority Acquisitions | Toehold 

	\begin{enumerate}
		\item Most firms, private and public, have assets on their books that can be considered to be non-operating assets. The second is investments in equities and bonds of other firms, sometimes for investment reasons and sometimes for strategic ones \citep{Damodaran2005}

		\item Minority acquisitions tend to involve the transfer of a sizable portion of the target firm, with a mean purchase of 12\%. \citep{Ouimet2013}
		\item Block ownership by corporations is unique relative to block ownership by institutions or individuals because of the possibility that business agreements, alliances, or joint ventures might be reached between target firms and corporate owners. \citep{Allen2000}
		\item Blockholdings -- First, block ownership might be useful in aligning the incentives of the firms involved in alliances or joint ventures \citep{Allen2000}
		\item Allen and Phillips (2000) and Fee, Hadlock, andThomas (2006) show that minority acquisitions can mitigate incomplete contracts and thereby facilitate cooperation between two independent firms \citep{Ouimet2013}
		\item The underlying logic is that such deals could both help the bidder to enforce incomplete contracts and to gather more information before launching a bid for full control. (huang)
		\item Acquire a minority position to access greater information about the target firm and better assess the potential for a majority acquisition. \citep{Ouimet2013}
		\item Our results suggest that bidders use minority acquisitions when they confront informational or integration barriers.(huang)
        \item Minority acquisitions can also help to serve as a stepping-stone towards full control. (huang)
        \item We show that minority stakes are also useful in mitigating some of the risks likely to affect takeover deals that involve greater information asymmetry (huang)

		\item Toehold: purchasing an ownership interest in a target firm prior to initiating merger-and-acquisitions discussions - threshold of owned shares is debatable 
		\item A toehold is defined as the raider’s ownership stake in the target firm prior to announcing his tender offer. SEC regulation specifies that anyone who acquires 5\%of a company’s outstanding shares must file Schedule 13(D) within ten days to disclose their identity, the number of shares owned, and their purpose in acquiring the shares. In practice, a bidder can continue purchasing target shares anonymously after hitting the 5\% threshold until the disclosure date. \citep{Goldman2005}
		\item A firm contemplating making a bid for the target may also decide to purchase target shares -- a toehold -- in the market at the pre-bid (no-information) target share price. \citep{Mitchell2011}
		\item toeholds are much more common in hostile than in friendly takeovers. While 11\% of initial bidders have toehold when the target is friendly, 50\% of the initial bidders in hostile contests have toeholds \citep{Mitchell2011}
		\item A toehold provides an opportunity to interact with the target and its management and in the process get a better sense of the possible synergies from a merger or takeover. \citep{Povel2014}
		\item The case for acquiring a toehold before initiating a takeover bid is compelling. The toehold not only reduces the number of shares that must be purchased at the full takeover premium, it may also be sold at an even greater premium should a rival bidder enter the contest and win the target. \citep{Eckbo2009}
		\item explore using the Capital IQ data, but it would be an interesting question for future research. We make some suggestions: A toehold often likely opens the door to a more intensive cooperation between the toeholder and the target, such that the toeholder learns more about the target's operations and prospects than regular suppliers or customers could hope to discover (without a toehold). That may happen because the toeholder negotiates the right to nominate one or more directors, who have direct access to the target's executives (a non-toeholder would not have that option), or through cooperation at the level of operations (by sharing production facilities or distribution networks). \citep{Povel2014}
		\item Moreover, the sample-wide frequency of short-term toeholds—defined as target shares purchased within six months of the offer—is only 2\%. In sum, toehold benefits notwithstanding, toeholds acquired as part of an active bidding strategy are almost nonexistent. \citep{Mitchell2011}
	\end{enumerate}